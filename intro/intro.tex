\chapter{Introduction}
\thispagestyle{plain}

Chemistry has given many blessings to mankind, providing drug molecules, plastics and other materials \cite{napoleons_buttons}. However, the golden age of chemistry, where new promising molecules can easily be found from natural products, or extracted from petroleum compounds is coming to an end. We have picked all the low hanging fruit of easy-to-find, easy to synthesize compounds, and for a sustainable, carbon neutral future, we must quell our dependence on petroleum.
In order to discover new molecules in the 21st century, we must rapidly increase our throughput for molecular discovery. Machine learning can help us arrive at this destination. Models using machine learning techniques have already been used to optimize tasks such as playing go and teaching robots and drones to walk. Machine learning models are even beginning to tackle creative tasks such as producing art and music. 
It is not only possible, but critical for machine learning models to be developed towards discovering new molecules, and optimizing new reactions.
There are a few issues facing the application of machine learning to chemistry. The first issue is that while there are many of sources of data (from hundreds of years of experiments), there is no central repository for accessing all of the data. Many of these records are not available to the public. Chemistry publications reporting new results and new reactions, typically only report the only the most successful molecule or reaction. The results do not include negative results. More complete datasets exist at pharmaceutical companies or in other industries, however, these datasets are treated as  proprietary. There are a few public datasets of molecules and reactions available, as well as other datasets that are available for sale. These datasets enable us to do some machine learning but come with some limitations. I will discuss this further in the Background section.
The second issue is that the challenges we wish to address in chemistry are not easy to formulate. Often, decisions in chemistry involve trade offs. Making molecules with a high reduction potential for use in flow battery may also cause them to be more susceptible to degradation reactions.\cite{tabor_2018} Choosing to improve the yield of a synthesis product by using fewer reaction steps will likely necessitate starting with more complicated materials, or more toxic reagents. Conversely, if you choose to use cheaper starting materials, you might have a long synthesis path, requiring multiple purification steps. The choice of how to how to prioritize these factors is subjective. The optimization should depend on the particular application, as well as the experiences of the chemists involved.
The third issue is the challenge of molecular representability. When a human chemist sees a molecule, they infer a lot of things about the molecule, including the electronegativity of regions of the molecule, steric hindrance, etc. The maxim taught in organic chemistry is “structure dictates function.” Because the molecule is a graph structure, it is difficult to encapsulate this information into a single vector for a molecular input.
In my doctoral research, I developed machine learning models for three applications in chemistry. I worked with the representations that currently exist to predict reactions and mass spectrometry. I have used machine learning models to compress the molecular space into a new representation for easier optimization and new lead selection.
This thesis is divided up into two main parts. In the first part, I provide some background for planning machine learning projects in chemistry. This section covers:
\begin{itemize}
\item Provide a heavily abbreviated introduction to machine learning, focusing in particular on models used in my machine learning models.
\item Consider some of the factors behind choosing an experimental dataset.
\item Discuss various representations for molecules for use in machine learning models.
\end{itemize}
The remainder of this thesis is divided into three different parts, each describing a different application for machine learning models in chemistry: 
\begin{itemize}
\item Predicting organic chemistry reactions using neural networks and molecular fingerprints
\item Predicting mass spectrometry spectra for small molecules using neural networks and fingerprint representations.
\item Using reversible, data-driven representations to encode molecules and use these representations to drive optimization for particular properties
\end{itemize}
I will conclude with an outlook on how to improve machine learning models for chemistry application, both in terms of modeling and dataset collection.

