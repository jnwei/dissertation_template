\chapter{Future Directions}\label{ch:future_directions}
\dsp

Machine learning has already begun to revolutionize the development of new materials.
Recent successes in machine learning have lead to the development of novel blue OLED molecules\cite{bombarelli2016}.

The methods that I present in Section \ref{sec:MLinChem} are a collection of early works into the next generation of this direction.
New molecules can be proposed from generative model, similar to those proposed in Chapter \ref{sec:autoencoder}.
The reactivity of these molecules can be predicted with the methods presented in Chapter \ref{sec:reaction_prediction}.
Newly synthesized molecules can be verified by with spectroscopy;
 machine learning can be used to aid identification by expanding the coverage of existing libraries, as shown
 in Chapter \ref{sec:massspec}.


I would like to close my dissertations with three main areas I see for improvement in the development of machine learning models for chemistry applications.

\textbf{Better datasets and benchmarks for testing machine learning models.}
Standardized, publicly available datasets are needed for the community as the whole to develop new models to push the capacity of machine learning models.
Datasets for molecules are available, and contain molecules which reflect molecules that are commonly used for drugs.
Datasets for reactions are much more limited. There is one publicly available dataset for reactions, the USPTO dataset \cite{lowe2012extraction}.
However this dataset contains only successful reactions, with numerous issues with the standardization of the data.
Some works have published their datasets splits, which is very helpful for reproducing results.
Ideally however, we would have more datasets, which would include more details of the reaction conditions, as well as examples of non-working reactions.

Additionally, there is a need for additional benchmarks for comparing generative models.
It is difficult to measure the relative progress of generative models without some method of comparing the quality of these datasets.
At the time of writing, three works have recently been released towards this goal \cite{Polykovskiy2018MolecularSets,Preuer2018Frechet,brown2018guacamol}.


\textbf{Better molecular representations.}
The current representations for molecules described in Chapter \ref{sec:cheminf} have been successfully employed in machine learning models
to predict a wide range of properties in chemistry.
However, there are some issues with this representations.
The SMILES representation has a few issues when combined with the variational autoencoder.
The purpose of a variational autoencoder is to group similar objects close together in the latent space.
However, SMILES strings that might be similar in terms of edit distances may not actually be close in molecular space:
\textit{c1ccccc1} (benzene) will have very different properties from \textit{c1ccccn1} (pyridine).
Additionally, even when grammar restraints are incorporated into the generation of the molecule string\cite{kusner2017grammar},
the model does not have a sense of which molecules are feasible in terms of their valence, and which molecules are not.

It is therefore necessary to have models which can generate graphs from a vector representation.
Several papers have been developed towards this direction at the time of writing
\cite{Ma2018Constrained,Liu2018Constrained,You2018GraphConvPolicy,jin2018junction}.

However, none of these representations are invariant to graph isomorphism. That is, a molecule which was
formed starting from one atom will be considered different from a molecule that was formed starting from another atom.
While it is possible to train these representations to be equivalent, it may lead to inefficiencies in the model.
While there are several methods such as graph convolutional networks for encoding from a graph,
there are limited methods for decoding to a molecule based representation.

\textbf{Better communication between machine learning experts and chemists.}
As demonstrated in the mass spectrometry project of Chapter \ref{sec:massspec},
the best models arise when the design of the model is inspired by the physical characteristics of the problems.
By reparametrizing the output from the neural network to account for larger fragments, we were able to improve the
performance of spectral reconstructions significantly.
Such design decisions can only be reached when there is constant dialogue between the two domains.
That way, we can ensure that the models that are developed in the machine learning community are both as accurate
and feasible as they can be.
This can also help tailor existing models to the needs of individual groups and projects.

With the development of new machine learning models and further developments in chemistry,
it will be possible to further accelerate the discovery and development of new molecular materials.
