\chapter{Future Directions}\label{ch:future_directions}
\dsp

The ultimate goal of all of the three models I have presented in this dissertation is to combine them all towards the development of new molecules. Variational autoencoders and other generative models can generate new molecule candidates for particular applications. Reaction prediction algorithms can be used to help screen out molecules which might be too reactive in their environment. Mass spectrometry prediction can be used by robotic workflows to verify the quantity of their content. 

Each of these model would likely have better performance with more sophisticated molecule representations. Depending on the task, however, such representations might be more difficult to use and train. For both the reaction prediction and the mass spectrometry spectra prediction projects, we tested using graph convolutional networks. However, due to the increased number of parameters from the graph convolutional network itself, it became much more difficult to train the parameters so that the model did not overfit the solution. 

With the autoencoder project, a molecule representation which encodes the molecular graph information as a set of components rather than as a SMILES string will likely see better reconstructive accuracy. There are several fundamental issues with using a string representation for a molecule rather than a graph representation. The purpose of the variational autoencoder is to group similar objects close together in the latent space. However, SMILES strings that might be similar in terms of edit distances may not actually be close in molecular space: \textit{c1ccccc1} (benzene) will have very different properties from \textit{c1ccccn1} (pyridine). Additionally, even when grammar restraints are incorporated into the generation of the molecule string\cite{kusner2017grammar}, the model does not have a sense of which molecules are feasible in terms of their valence, and which molecules are not.

While there are several methods such as graph convolutional networks for encoding from a graph, there are limited methods for decoding to a molecule based representation. One recently developed approach towards this goal is the Junction Tree Variational Autoencoder\cite{jin2018junction}. This modeling method records the molecule as a collection of pre-defined functional groups, enabling decoded vectors to be translated directly into molecular graphs. 

I would like to close my dissertations with three main areas I see for improvement in the development of machine learning models for chemistry applications.

\begin{enumerate}

\item Better model distribution and benchmarks. If there existed a centralized platform for distributing new models, and a standardized way of inputing molecules, this would make comparison across molecules much easier. Right now a lot of time is spent on trying reconstruct the errors of others.

\item More standardized datasets. Experimentalists should be rewarded for sharing their data into standardized formats for machine learning. One recent effort is this NIH collaboration, which was a collabration across 20 different labs to gather some biology data. Efforts such as this make it possible for everyone to train on the same datasets and compare models directly, so that the best models can be identified. The current datasets that exist are often noisy and had to be parsed out of existing records, such as the USPTO reaction dataset.

\item Better communication between machine learning experts and chemists. As demonstrated in the mass spectrometry project of Chapter 3, the best models arise when the design of the model is inspired by the physical characteristics of the problems. By accounting for the presence of ion fragments that are the residuals ion fragments of common fragmentation patterns, we were able to improve the model accuracy there considerably.

\end{enumerate}

Such design decisions can only be reached when there is constant dialogue between the two domains. That way, we can ensure that the models that are developed in the machine learning community are both as accurate and feasible as they can be.

This can also help tailor models to the needs of individual groups and projects.

